\documentclass[a4paper,11pt,twoside]{article}
%\documentclass[a4paper,11pt,twoside,se]{article}

\usepackage{UmUStudentReport}
\usepackage{verbatim}   % Multi-line comments using \begin{comment}
\usepackage{courier}    % Nicer fonts are used. (not necessary)
\usepackage{pslatex}    % Also nicer fonts. (not necessary)
\usepackage[pdftex]{graphicx}   % allows including pdf figures
\usepackage{listings}
%\usepackage{lmodern}   % Optional fonts. (not necessary)
%\usepackage{tabularx}
%\usepackage{microtype} % Provides some typographic improvements over default settings
%\usepackage{placeins}  % For aligning images with \FloatBarrier
%\usepackage{booktabs}  % For nice-looking tables
%\usepackage{titlesec}  % More granular control of sections.

% DOCUMENT INFO
% =============
\department{Institution för Datavetenskap}
\coursename{Datavetenskapens byggstenar 7.5 p}
\coursecode{DV160HT15}
\title{OU4 Data Representation}
\author{Lorenz Gerber ({\tt{dv15lgr@cs.umu.se}})} 
\date{2016-03-03}
%\revisiondate{2016-01-18}
\instructor{Lena Kallin Westin / Johan Eliasson / Emil Marklund / Lina
Ögren}


% DOCUMENT SETTINGS
% =================
\bibliographystyle{plain}
%\bibliographystyle{ieee}
\pagestyle{fancy}
\raggedbottom
\setcounter{secnumdepth}{2}
\setcounter{tocdepth}{2}
%\graphicspath{{images/}}   %Path for images

\usepackage{float}
\floatstyle{ruled}
\newfloat{listing}{thp}{lop}
\floatname{listing}{Listing}


% DEFINES
% =======
%\newcommand{\mycommand}{<latex code>}

% DOCUMENT
% ========
\begin{document}
\lstset{language=C}
\maketitle

\tableofcontents
\newpage

\section{Introduction} 
In this assignment the aim was to specify three different possible
data representations for a spreadsheet application. The
representations were to be described such that they could be implemented
from the descriptions.
 
Several criteria to judge the suitability of the chosen representations were 
discussed in the mandatory seminar \emph{OU2}. Three of those criteria 
were then applied to judge the chosen data representations.


%   \begin{quote}
% Imagine that you work with the development of a spreadsheet software
% (Excel). The spreadsheet is basically a single large table with rows
% and columns. Each cell has a unique address in the form of a row and a
% column coordinate. For simplicity, let's represent both of them as an
% integer (normally columns are represented by character combinations,
% but the transformation to and from is trivial). The table is basically 
% of infinite size and the user fills in just some cells here and
% there. Obviously, it's not possible to represent all cells as they are
% of infinite count. Neither is it very economic to to represent the
% finit number of cells that fit in the thought minimum rectangle
% that includes all non-empty cells as still many included cells can be
% empty. The aim is to find a more economic representation of the
% spreadsheet. There are potentially many different viable
% representations. The assignment is to come up with three different
% representations for a spreadsheet and discuss pros and cons for each
% of them. However the first step is to determine suitable such criteria
% for evaluating the represantation.
%     \end{quote}


\subsection{Interpretation of the Problem Descrition}
The problem description from the course homepage defines a spreadsheet
as a 'table'. Hence, a spreadsheet can be seen seen as a potentially
infinite table. A requirement given in the description is that the
implementation shall be more economic than an plain rectangular
structure covering all non-empty table elements.

\subsection{Typical Usecases of a Spreadsheet}
During a user session, text, numbers, formulas and links to
other elements are stored in the table elements. In many usecases the
number of filled elements will be very low compared to the virtual
rectanglular set of elements that surrounds the outermost non-empty
table elements, hence the table is said to have a low \emph{fill
 ratio}. This is the main reason why the potential data structure
should only store non-empty elements.

Another important property of a spreadsheet program is that the data
structure is represented in the graphical user interface. Due to the
size, there will usually just one part of the data be visible, hence
blockwise value lookup for scrolling over the data table is a very
common operation.

Spreadsheets are often used to prepare sorted lists. Hence
rearranging the order of whole rows or columns is another operation of
high importance.

When spreadsheets are used for calculation purposes, extensive linking
between memeber elements of the spreadsheet is common. Links are
either between two elements, much in the style of a pointer, but then
can also be many elements to one cell, in the case of functions, such
as calculating the sum of multiple cells.




\subsection{Chosen Criteria}
\subsubsection{Time Complexity}
The speed of specific operations is an important criteria. In some
cases, it could decide whether a ceratin construction is feasible at
or not. Whith `speed' I would define the processing time needed at a
realistic use case size of the instance in question. Hence, sometimes
a bad complexity can be accepted as the typical realworld instance
size happens to be in a very narrow bandwith.
If possible, also the time complexity of individual operations will be
asessed.    

The chosen operations of interest are given below:

\begin{description}
\item[Block Lookup] Get values for a block of cells. This operation
  will usally be applied when the user scrolls or jumps to another
  place in the spreadsheet. If row/column wise scrolling is assumed,
  typicall lookup sizes would be around 10 to 50. 
\item[Link following] Lookup of links as used in functions. 
\item[Deletion] Deleting elements of the spreadsheet.
\item[Traversal] Traversing the data structure to search for a value
\end{description}

\subsubsection{Ease of Implementation}
During OU2 discussion, it was agreed that a lower complexity
of the implementation is generally desirable as it decreases the
susceptibility for bugs and code maintenance. Below are indicated some
features/operations for which the ease of implementation will be
judged and compared across the different table constructions.

\begin{description}
\item[Basic Structure] The data container as such
\item[Sort] Sorting larger blocks of non-empty elements
\end{description}

\subsubsection{Memory efficency}
It is understood from the problem definition that the used
construction shall just store non-empty spreadsheet elements. Hence
the judgement of memory efficency focuses on the amount of
memory used for `administratiion' of the non-empty elements, such as
hash tables and non value bearing data structure. 


\subsection{Chosen Datatypes}
\subsubsection{table}
\emph{Table} was the most obvious choice. More specific, the
construction of the table has to be such that it can grow
and shrink dynamically. This is true for a table constructed from a
linked list. 

\subsubsection{Binary trees}
Binary search trees offer a good time complexity. A table can be
implemented as one tree for the rows and multiple trees for the
columns in each row. 

\subsubsection{Directed Acyclic Graph and Hash Table}
A directed graph is an n-linked structure. The nodes can be
constructed as cells. The cells are created dynamically on demand. 
Edges represent links in the spreadsheet. They are also created 
dynamically. Direct access to individual nodes is achieved by a hash
table. 

 

\section{Material and Methods}
\Subsection{Representation as Table}
For the table implementation, the specifications given in Janlert (2000)
\cite[p 124]{janlert2001} are suggested. 

\section{Results}
And our results looked like this...

\section{Discussion}

bla bla bla... 

\addcontentsline{toc}{section}{\refname}
\bibliography{references}

\end{document}
