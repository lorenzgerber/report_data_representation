\documentclass[a4paper,11pt,twoside]{article}
%\documentclass[a4paper,11pt,twoside,se]{article}

\usepackage{UmUStudentReport}
\usepackage{verbatim}   % Multi-line comments using \begin{comment}
\usepackage{courier}    % Nicer fonts are used. (not necessary)
\usepackage{pslatex}    % Also nicer fonts. (not necessary)
\usepackage[pdftex]{graphicx}   % allows including pdf figures
\usepackage{listings}
%\usepackage{lmodern}   % Optional fonts. (not necessary)
%\usepackage{tabularx}
%\usepackage{microtype} % Provides some typographic improvements over default settings
%\usepackage{placeins}  % For aligning images with \FloatBarrier
%\usepackage{booktabs}  % For nice-looking tables
%\usepackage{titlesec}  % More granular control of sections.

% DOCUMENT INFO
% =============
\department{Institution för Datavetenskap}
\coursename{Datavetenskapens byggstenar 7.5 p}
\coursecode{DV160HT15}
\title{OU4 Data Representation}
\author{Lorenz Gerber ({\tt{dv15lgr@cs.umu.se}})} 
\date{2016-03-03}
%\revisiondate{2016-01-18}
\instructor{Lena Kallin Westin / Johan Eliasson / Emil Marklund / Lina
Ögren}


% DOCUMENT SETTINGS
% =================
\bibliographystyle{plain}
%\bibliographystyle{ieee}
\pagestyle{fancy}
\raggedbottom
\setcounter{secnumdepth}{2}
\setcounter{tocdepth}{2}
%\graphicspath{{images/}}   %Path for images

\usepackage{float}
\floatstyle{ruled}
\newfloat{listing}{thp}{lop}
\floatname{listing}{Listing}


% DEFINES
% =======
%\newcommand{\mycommand}{<latex code>}

% DOCUMENT
% ========
\begin{document}
\lstset{language=C}
\maketitle

\tableofcontents
\newpage

\section{Introduction} 
In this assignment the aim was to specify three different possible
data representations for a spreadsheet application. The
representations were to be described such that they could be implemented
from the descriptions.
 
Several criteria to judge the suitability of the chosen representations were 
discussed in the mandatory seminar \emph{OU2}. Three of those criteria 
were then to be used to judge the chosen data representations.

\subsection{Problem Description}
Below is a translation of the the problem description given in Swedish
on the course homepage:

\begin{quote}
Imagine that you work with the development of a spreadsheet software
(Excel). The spreadsheet is basically a single large table with rows
and columns. Each cell has a unique address in the form of a row and a
column coordinate. For simplicity, let's represent both of them as an
integer (normally columns are represented by character combinations,
but the transformation to and from is trivial). The table is basically 
of infinite size and the user fills in just some cells here and
there. Obviously, it's not possible to represent all cells as they are
of infinite count. Neither is it very economic to to represent the
finit number of cells that fit in the thought minimum rectangle
that includes all non-empty cells as still many included cells can be
empty. The aim is to find a more economic representation of the
spreadsheet. There are potentially many different viable
representations. The assignment is to come up with three different
representations for a spreadsheet and discuss pros and cons for each
of them. However the first step is to determine suitable such criteria
for evaluating the represantation.
\end{quote}

I chose to structure the report according a scientific article
with \emph{Introduction}, \emph{Material and Methods}, \emph{Results},
\emph{Discussion} and \emph{References}. 

\subsection{The Anatomy of a Spreadsheet}
A spreadsheet can be described as a matrix, two-dimensional array or a
table with polytypic elements: Each element allowa storage of various
simple data types. 

\subsection{Typical Usecases of a Spreadsheet}
During a typical user session, text, numbers,
formulas and links to other elements are stored in the table/array
elements. In many usecases the array 


\subsection{Fast and Large or Slow and Small}
The most general criteria applicable for in most cases of data
structures are \emph{Time} and \emph{Space}. Most design choices come
down to a trade-off between how execution speed and amount of needed
memory. 



\section{Material and Methods}
DAG, directed acyclic graph. 

\section{Results}
And our results looked like this...

\section{Discussion}

bla bla bla... 

\addcontentsline{toc}{section}{\refname}
\bibliography{references}

\end{document}
